\documentclass[a4paper]{scrartcl}
\usepackage[utf8]{inputenc} % necessary for the BiBTeX file to work with umlauts
\usepackage[T1]{fontenc}
\usepackage{amsmath,amsfonts,amsthm}
\usepackage[ngerman]{babel}
\usepackage{enumerate}
\usepackage{natbib} % provides a few nice styles for the bibliography
\usepackage[nottoc]{tocbibind} % makes sure the bibliography is added to the table of contents
\usepackage[fixlanguage]{babelbib}
\selectbiblanguage{german} % selects German as language for the bibliography
\usepackage{graphics}

\newcommand{\C}{{\mathbb{C}}}
\newcommand{\R}{{\mathbb{R}}}
\newcommand{\Z}{{\mathbb{Z}}}
\newcommand{\de}{{\mathrm{d}}}
\newcommand{\ee}{{\mathrm{e}}}
\newcommand{\ii}{{\mathrm{i}}}
\newcommand{\norm}[1]{{\left\lVert#1\right\rVert}}
\newcommand{\pphi}{{\varphi}}
\newcommand{\defeq}{\overset{!}{=}}
\newcommand{\todo}[1]{{TODO: {#1}}}


\begin{document}

%\tableofcontents

\title{Zerlegung von univariaten Signalen in amplituden- und frequenzmodulierte IMF-Komponenten}
\author{Ralph Tandetzky}
\date{27. Juni 2014}
\maketitle

\section{Die Problemstellung}

H"aufig versucht man EEG- und MEG-Signale oder andere Zeitreihen zu untersuchen, indem man sie in eine "Uberlagerung von IMFs (intrinsic mode functions) aufteilt. 
Die IMFs variieren in Amplitude und Frequenz und k"onnen in der Form~$a(t)\cos\pphi(t)$ dargestellt werden. 
Ist ein Signal~$f:[a,b]\to\R$ gegeben, so ist die Aufgabe, eine Zerlegung der Form 
$$ f(t) = \sum_{i=1}^N a_i(t)\cos\pphi_i(t) + r(t) $$
zu finden, wobei die Amplituden~$a_i$, die Phasen~$\pphi_i$ und die Restfunktion~$r$ reellwertige Funktionen auf dem Intervall~$[a,b]$ sein sollen und~$r$ minimiert werden soll unter der Nebenbedingung, dass die Summanden~$a_i(t)\cos\pphi_i(t)$ gutartige IMFs sind. 
Zu kl"aren, was \glqq{}gutartig\grqq{} hei"st, ist Teil der Problemstellung. 


\section{Die Grundidee}

Es wird stets davon ausgegangen, dass $a_i(t)>0$ und $\dot\pphi_i(t)>0$ ist. 
Diese Bedingungen reichen jedoch nicht, um zu garantieren, dass die Summanden IMFs sind, also nicht mehrere Extrema zwischen zwei Nullstellen haben k"onnen. 
Wenn der Logarithmus des Betrags eines Summanden zwischen je zwei Nullstellen konkav (nach unten gebogen) ist, dann ist jedoch die IMF-Eigenschaft sichergestellt. 
Dazu betrachten wir die zweite Ableitung des Betrags des Summanden~$a(t)\cos\pphi(t)$. 
(Wir lassen den Index $i$ f"ur diese Rechnung weg.)
Diese hat die Form
$$ \frac{\de^2}{\de t^2}\ln a(t)
-\ddot\pphi(t)\tan\pphi(t)
-\frac{\dot\pphi(t)^2}{\cos^2\pphi(t)} $$
und muss nicht-positiv sein, um zu erreichen, dass der Betrag des Summanden zwischen den Nullstellen logarithmisch konkav\footnote{Die Eigenschaft, logarithmisch konkav zu sein, ist echt schw"acher als die Eigenschaft, konkav zu sein.} ist. 
Wir definieren $\alpha(t):=\ln a(t)$. 
Der obige Ausdruck multipliziert mit $\cos^2\pphi(t)$ ergibt
$$ \ddot\alpha(t)\cos^2\pphi(t)
-\frac12\ddot\pphi(t)\sin2\pphi(t)
-\dot\pphi(t)^2
\le 0\,. $$
Dies ist "aquivalent zu
$$ \frac12\ddot\alpha(t)\cos2\pphi(t)
-\frac12\ddot\pphi(t)\sin2\pphi(t)
\le \dot\pphi(t)^2-\frac12\ddot\alpha(t)\,. $$
Die linke Seite ist gleich dem Skalarprodukt der beiden $\R^2$-Vektoren $(\cos2\pphi(t);\sin2\pphi(t))$ und $\frac12(\ddot\alpha(t);\ddot\pphi(t))$ und somit gem"a"s der Cauchy-Schwarzschen Ungleichung jedenfalls nicht gr"o"ser als das Produkt der euklidischen Normen beider Vektoren. Also ist diese Bedingung erf"ullt, falls 
$$ \frac12\sqrt{\ddot\alpha(t)^2+\ddot\pphi(t)^2}
\le \dot\pphi(t)^2-\frac12\ddot\alpha(t) $$
gilt.\footnote{Dieser Schritt bei der Absch"atzung bewirkt, dass die Nebenbedingung unabh"angig von einer additiven Konstante wird, die man zu $\pphi(t)$ hinzuf"ugen k"onnte. 
Man k"onnte also im Nachhinein die Phase um $90^\circ$ drehen, ohne dadurch etwas an der G"ultigkeit dieser Nebenbedingungen zu "andern.
Damit kann also aus einem reellen Signal ein komplexes Signal erzeugt werden, indem man ein um $90^\circ$ gedrehtes Signal als Imagin"arteil hinzuf"ugt. 
Dies stellt eine Alternative zur Hilbert-Transformation dar mit dem Vorteil, dass die Phasengeschwindigkeit stets positiv ist.
}
Und diese Bedingung ist wiederum erf"ullt, wenn
$$ \sqrt{\ddot\alpha(t)^2+\ddot\pphi(t)^2}
\le \dot\pphi(t)^2 $$
gilt. Wir haben also gezeigt, dass diese letzte Ungleichung zusammen mit $a(t)>0$ und $\dot\pphi(t)>0$ eine hinreichende Bedingung daf"ur ist, dass $a(t)\cos\pphi(t)$ eine IMF ist. 

Gleichzeitig ist es aber auch eine Bedingung an die Glattheit der IMF, also dass sich der Summand \glqq{}gutartig\grqq{} verh"alt.
Das ist die Grundidee der gefundenen Methode. 
Es wird im Folgenden kurz beschrieben, wie passende Funktionen gefunden werden k"onnen, die diese Nebenbedingung erf"ullen. 

\section{Umsetzung der Optimierung}

F"ur die Optimierung wurden der Amplitudenlogarithmus $\alpha(t)$ und die Phasenfunktion $\pphi(t)$ wurden durch eine Radialbasis bzw. eine Basis aus logistischen Funktionen dargestellt. 
Die Koeffizienten wurden anhand des {\em Differential Evolution} Algorithmus~\citep*{StPr1997} so optimiert, dass eine m"oglichst kleine Restfunktion~$r(t)$ entsteht. 
Die Nebenbedingungen werden durch entsprechende Strafterme in der Kostenfunktion erzwungen. 

Au"serdem werden die IMFs iterativ optimiert. 
Dies geschieht, indem zuerst $a_1(t)$ und $\pphi_1(t)$ optimiert werden, wobei die anderen Summanden vernachl"assigt werden. 
Dann werden diese beiden Funktionen fixiert und derselbe Algorithmus auf den Rest $f(t) - a_1(t)\cos\pphi_1(t)$ angewendet, um $a_2(t)$ und $\pphi_2(t)$ zu bestimmen. 
So wird fortgefahren, bis alle IMFs berechnet worden sind. 

\section{Vergleich mit anderen Methoden}

Eine der besten bekannten Methoden, um ein Signal in IMFs zu zerlegen ist die sogenannte {\em Complete Ensemble Empirical Mode Decomposition (cEEMD)}~\citep*{TCSF2011}.
Ein Problem bei allen dem Autor bekannten Methoden ist das Mode Mixing. 

\todo{Konkrete Ergebnisse von EEG-Testdaten vergleichen}


\section{Weitere Arbeit} 

Es gibt noch folgende Herausforderungen f"ur die Zukunft:
\begin{itemize}
\item Die Methode funktioniert sehr gut auf kurzen Intervallen. 
Bei langen Zeitintervallen werden allerdings oft schlechte Ergebnisse gefunden, da die notwendige Zahl der zu fittenden Parameter zu gro"s wird. 
\item Die Optimierung braucht vergleichsweise viel Zeit.
\item Die Methode zeigt, dass nur sehr wenige IMFs gebraucht werden, um die Daten so zu beschreiben, dass als Restfunktion nur noch Rauschen "ubrig bleibt. 
F"ur die Zukunft sollte es m"oglich sein, dass der Algorithmus selbst entscheidet, wann abgebrochen wird. 
\end{itemize}
Um auf l"angeren Intervallen erfolgreich zu optimieren, gibt es folgende Ans"atze:
\begin{itemize}
\item Den Differential Evolution Algorithmus so modifizieren, dass immer nur Teilintervalle optimiert werden und der Rest der IMFs gleich bleibt. 
  Dann wird das Teilintervall zuf"allig gewechselt und das neue Teilintervall weiter optimiert. 
\item Statt Radialbasen und logistischen Funktionen k"onnten quadratische B-Splines als Basis verwendet werden. 
  Diese haben einen kompakten Tr"ager und die Koeffizienten wirken sich daher nur lokal aus. 
  Deswegen kann diese Idee kann Grundlage f"ur die angesprochene Modifikation des Differential Evolution Algorithmus benutzt werden. 
\item Die B-Splines haben au"serdem den Vorteil, dass sie leicht mit dem Faktor 2 skalierbar sind. 
  Das hei"st, man k"onnte anfangen sehr grob mit wenigen Parametern zu optimieren. 
  Sp"ater wird die Feinheit der B-Splines (und somit die Anzahl der Parameter) schrittweise verdoppelt, um genauere Ergebnisse zu erlangen. 
  Dies d"urfte auch die notwendige Optimierungszeit massiv verringern. 
\end{itemize}


%\includegraphics{decompose_imf-1.mps}


% Ideen
% =====
%
% * Radialbasen und logistische Basen durch Basen quadratischer B-Splines ersetzen. 
% * Multigrid-Verfahren: Der Zahl der Knotenpunkte der B-Splines kann schrittweise verdoppelt werden. 
% * Reduktion der zu fittenden Parameter durch analytische Ermittelung von Phasenoffset, Amplitudenoffset und Momenten der gefitteten Funktion. 
% * Phasenboost implementieren. 
% * Bei den B-Splines k"onnte ein Intervall ausgew"ahlt werden, auf dem Elemente des Schwarms kombiniert werden. 



\bibliographystyle{plainnat}
\bibliography{decompose_imf}


\end{document}
