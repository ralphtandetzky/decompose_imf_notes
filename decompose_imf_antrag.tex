\documentclass[a4paper]{scrartcl}
\usepackage[utf8]{inputenc} % necessary for the BiBTeX file to work with umlauts
\usepackage[T1]{fontenc}
\usepackage{amsmath,amsfonts,amsthm}
\usepackage[ngerman]{babel}
\usepackage{enumerate}
\usepackage{natbib} % provides a few nice styles for the bibliography
\usepackage[nottoc]{tocbibind} % makes sure the bibliography is added to the table of contents
\usepackage[fixlanguage]{babelbib}
\selectbiblanguage{german} % selects German as language for the bibliography

\newcommand{\C}{{\mathbb{C}}}
\newcommand{\R}{{\mathbb{R}}}
\newcommand{\Z}{{\mathbb{Z}}}
\newcommand{\de}{{\mathrm{d}}}
\newcommand{\ee}{{\mathrm{e}}}
\newcommand{\ii}{{\mathrm{i}}}
\newcommand{\norm}[1]{{\left\lVert#1\right\rVert}}
\newcommand{\pphi}{{\varphi}}
\newcommand{\defeq}{\overset{!}{=}}


\begin{document}

%\tableofcontents

\title{Zerlegung von univariaten Signalen in amplituden- und frequenzmodulierte IMF-Komponenten}
\author{Ralph Tandetzky}
\date{20. Juni 2014}
\maketitle

\section{Die Problemstellung}

H"aufig versucht man EEG- und MEG-Signale oder andere Zeitreihen zu untersuchen, indem man sie in eine "Uberlagerung von IMFs (intrinsic mode functions) aufteilt. 
Die IMFs variieren in Amplitude und Frequenz und k"onnen in der Form~$a(t)\cos\pphi(t)$ dargestellt werden. 
Ist ein Signal~$f:[a,b]\to\R$ gegeben, so ist die Aufgabe, eine Zerlegung der Form 
$$ f(t) = \sum_{i=1}^N a_i(t)\pphi_i(t) + r(t) $$
zu finden, wobei die Amplituden~$a_i$, die Phasen~$\pphi_i$ und die Restfunktion~$r$ reellwertige Funktionen auf dem Intervall~$[a,b]$ sein sollen und~$r$ minimiert werden soll unter der Nebenbedingung, dass die Summanden~$a_i(t)\pphi_i(t)$ gutartige IMFs sind. 
Zu kl"aren, was \glqq{}gutartig\grqq{} hei"st, ist Teil der Problemstellung. 


\section{Der Stand der Entwicklung}

Um gutartige L"osungen zu erlangen, wurden plausible Nebenbedingungen erarbeitet, die sich durch Ungleichungen der Ableitungen von Amplituden- und Phasenfunktion schreiben lassen. 
Der Amplitudenlogarithmus und die Phasenfunktion wurden durch eine Radialbasis bzw.~logistische Funktionen dargestellt und die Koeffizienten optimiert, um eine m"oglichst kleine Restfunktion~$r$ zu erhalten. 



% Ideen
% =====
%
% * Radialbasen und logistische Basen durch Basen quadratischer B-Splines ersetzen. 
% * Multigrid-Verfahren: Der Zahl der Knotenpunkte der B-Splines kann schrittweise verdoppelt werden. 
% * Reduktion der zu fittenden Parameter durch analytische Ermittelung von Phasenoffset, Amplitudenoffset und Momenten der gefitteten Funktion. 
% * Phasenboost implementieren. 
% * Bei den B-Splines k"onnte ein Intervall ausgew"ahlt werden, auf dem Elemente des Schwarms kombiniert werden. 



\bibliographystyle{plainnat}
\bibliography{decompose_imf}


\end{document}
