\documentclass[a4paper]{scrartcl}
\usepackage[utf8]{inputenc} % necessary for the BiBTeX file to work with umlauts
\usepackage[T1]{fontenc}
\usepackage{amsmath,amsfonts,amsthm}
\usepackage[ngerman]{babel}
\usepackage{enumerate}
\usepackage{natbib} % provides a few nice styles for the bibliography
\usepackage[nottoc]{tocbibind} % makes sure the bibliography is added to the table of contents
\usepackage[fixlanguage]{babelbib}
\selectbiblanguage{german} % selects german as language for the bibliography

\newcommand{\C}{{\mathbb{C}}}
\newcommand{\R}{{\mathbb{R}}}
\newcommand{\Z}{{\mathbb{Z}}}
\newcommand{\de}{{\mathrm{d}}}
\newcommand{\ii}{{\mathrm{i}}}
\newcommand{\norm}[1]{{\left\lVert#1\right\rVert}}
\newcommand{\pphi}{{\varphi}}
\newcommand{\defeq}{\overset{!}{=}}


\begin{document}

%\tableofcontents

\section{Zusammenfassung}

Es ist ein Signal $f:[a,b]\to\R$ gegeben, von dem wir annehmen, dass es durch reellwertige Funktion auf einem Intervall $[a,b]$ beschrieben wird. 
Gesucht ist eine Zerlegung
$$ f(t) = \sum_{i=1}^N a_i(t)\cos\pphi_i(t) + r(t)\,, $$
wobei $a_i,\pphi_i$ f"ur $i=1,\dotsc,N$ und $r$ reellwertige Funktionen auf dem Intervall $[a,b]$ sind und die Restfunktion $r(t)$ minimiert werden soll. 
Die Summe ist also eine "Uberlagerung von amplituden- und phasenmodulierten Signalen. 
Entsprechend sollen f"ur die Funktionen $a_i$ und $\pphi_i$ f"ur $i=1,\dotsc,N$ mindestens folgende Bedingungen f"ur alle $t$ gelten:
\begin{align*}
  a_i(t) &> 0 \\
  \dot{\pphi_i}(t) &\ge 0
\end{align*}
Au"serdem sollten die Summanden $a_i(t)\cos\pphi_i(t)$ zwischen je zwei benachbarten Nullstellen genau ein Extremum besitzen, so dass die Eigenschaft einer IMF (intrinsic mode function) erf"ullt ist. 
Die Aufgabe ist es nun, die $a_i(t)$ und $\pphi_i(t)$ numerisch so zu bestimmen, dass die Restfunktion $r(t)$ m"oglichst klein wird. 

%- bisherige Nebenbedingungen zu schwach => viele L"osungen, etc.
%- zus"atzliche Forderungen, um plausible L"osungen zu finden. 
%- Hinreichende Bedingungen in Form einer Differentialungleichung f"ur die IMF-Eigenschaft.
%- Diskretisierung mit B-Splines.
%- Zerlegung von Phase in logistische Basisfunktionen, und von Amplitute in radiale Basisfunktionen. 
%- F"ur die Optimierung brauch man eine Kostenfunktion, die sich zusammensetzt aus:
%  - einer Norm f"ur die Restfunktion und
%  - Strafterme f"ur die Verletzung der Nebenbedingungen.
%- F"ur die globale Optimierung braucht man
%  - eine geeignete Anfangsapproximation und
%  - einen globalen Optimierungsalgorithmus.
%- Vorverarbeitungsschritte: 
%  - Gl"attung der Funktion $f$, 
%  - alle Momente bis zur Ordnung $2$ auf null bringen.
%- Iterative Berechnung von IMF-Komponenten
%- Konkrete Ergebnisse in Testl"aufen
%- kurze Beschreibung der entstandenen Software



%\bibliographystyle{plainnat}
%\bibliography{decompose_imf}


\end{document}
